% Options for packages loaded elsewhere
\PassOptionsToPackage{unicode}{hyperref}
\PassOptionsToPackage{hyphens}{url}
%
\documentclass[
]{article}
\usepackage{lmodern}
\usepackage{amssymb,amsmath}
\usepackage{ifxetex,ifluatex}
\ifnum 0\ifxetex 1\fi\ifluatex 1\fi=0 % if pdftex
  \usepackage[T1]{fontenc}
  \usepackage[utf8]{inputenc}
  \usepackage{textcomp} % provide euro and other symbols
\else % if luatex or xetex
  \usepackage{unicode-math}
  \defaultfontfeatures{Scale=MatchLowercase}
  \defaultfontfeatures[\rmfamily]{Ligatures=TeX,Scale=1}
\fi
% Use upquote if available, for straight quotes in verbatim environments
\IfFileExists{upquote.sty}{\usepackage{upquote}}{}
\IfFileExists{microtype.sty}{% use microtype if available
  \usepackage[]{microtype}
  \UseMicrotypeSet[protrusion]{basicmath} % disable protrusion for tt fonts
}{}
\makeatletter
\@ifundefined{KOMAClassName}{% if non-KOMA class
  \IfFileExists{parskip.sty}{%
    \usepackage{parskip}
  }{% else
    \setlength{\parindent}{0pt}
    \setlength{\parskip}{6pt plus 2pt minus 1pt}}
}{% if KOMA class
  \KOMAoptions{parskip=half}}
\makeatother
\usepackage{xcolor}
\IfFileExists{xurl.sty}{\usepackage{xurl}}{} % add URL line breaks if available
\IfFileExists{bookmark.sty}{\usepackage{bookmark}}{\usepackage{hyperref}}
\hypersetup{
  pdftitle={R Notebook: Function},
  hidelinks,
  pdfcreator={LaTeX via pandoc}}
\urlstyle{same} % disable monospaced font for URLs
\usepackage[margin=1in]{geometry}
\usepackage{color}
\usepackage{fancyvrb}
\newcommand{\VerbBar}{|}
\newcommand{\VERB}{\Verb[commandchars=\\\{\}]}
\DefineVerbatimEnvironment{Highlighting}{Verbatim}{commandchars=\\\{\}}
% Add ',fontsize=\small' for more characters per line
\usepackage{framed}
\definecolor{shadecolor}{RGB}{248,248,248}
\newenvironment{Shaded}{\begin{snugshade}}{\end{snugshade}}
\newcommand{\AlertTok}[1]{\textcolor[rgb]{0.94,0.16,0.16}{#1}}
\newcommand{\AnnotationTok}[1]{\textcolor[rgb]{0.56,0.35,0.01}{\textbf{\textit{#1}}}}
\newcommand{\AttributeTok}[1]{\textcolor[rgb]{0.77,0.63,0.00}{#1}}
\newcommand{\BaseNTok}[1]{\textcolor[rgb]{0.00,0.00,0.81}{#1}}
\newcommand{\BuiltInTok}[1]{#1}
\newcommand{\CharTok}[1]{\textcolor[rgb]{0.31,0.60,0.02}{#1}}
\newcommand{\CommentTok}[1]{\textcolor[rgb]{0.56,0.35,0.01}{\textit{#1}}}
\newcommand{\CommentVarTok}[1]{\textcolor[rgb]{0.56,0.35,0.01}{\textbf{\textit{#1}}}}
\newcommand{\ConstantTok}[1]{\textcolor[rgb]{0.00,0.00,0.00}{#1}}
\newcommand{\ControlFlowTok}[1]{\textcolor[rgb]{0.13,0.29,0.53}{\textbf{#1}}}
\newcommand{\DataTypeTok}[1]{\textcolor[rgb]{0.13,0.29,0.53}{#1}}
\newcommand{\DecValTok}[1]{\textcolor[rgb]{0.00,0.00,0.81}{#1}}
\newcommand{\DocumentationTok}[1]{\textcolor[rgb]{0.56,0.35,0.01}{\textbf{\textit{#1}}}}
\newcommand{\ErrorTok}[1]{\textcolor[rgb]{0.64,0.00,0.00}{\textbf{#1}}}
\newcommand{\ExtensionTok}[1]{#1}
\newcommand{\FloatTok}[1]{\textcolor[rgb]{0.00,0.00,0.81}{#1}}
\newcommand{\FunctionTok}[1]{\textcolor[rgb]{0.00,0.00,0.00}{#1}}
\newcommand{\ImportTok}[1]{#1}
\newcommand{\InformationTok}[1]{\textcolor[rgb]{0.56,0.35,0.01}{\textbf{\textit{#1}}}}
\newcommand{\KeywordTok}[1]{\textcolor[rgb]{0.13,0.29,0.53}{\textbf{#1}}}
\newcommand{\NormalTok}[1]{#1}
\newcommand{\OperatorTok}[1]{\textcolor[rgb]{0.81,0.36,0.00}{\textbf{#1}}}
\newcommand{\OtherTok}[1]{\textcolor[rgb]{0.56,0.35,0.01}{#1}}
\newcommand{\PreprocessorTok}[1]{\textcolor[rgb]{0.56,0.35,0.01}{\textit{#1}}}
\newcommand{\RegionMarkerTok}[1]{#1}
\newcommand{\SpecialCharTok}[1]{\textcolor[rgb]{0.00,0.00,0.00}{#1}}
\newcommand{\SpecialStringTok}[1]{\textcolor[rgb]{0.31,0.60,0.02}{#1}}
\newcommand{\StringTok}[1]{\textcolor[rgb]{0.31,0.60,0.02}{#1}}
\newcommand{\VariableTok}[1]{\textcolor[rgb]{0.00,0.00,0.00}{#1}}
\newcommand{\VerbatimStringTok}[1]{\textcolor[rgb]{0.31,0.60,0.02}{#1}}
\newcommand{\WarningTok}[1]{\textcolor[rgb]{0.56,0.35,0.01}{\textbf{\textit{#1}}}}
\usepackage{graphicx,grffile}
\makeatletter
\def\maxwidth{\ifdim\Gin@nat@width>\linewidth\linewidth\else\Gin@nat@width\fi}
\def\maxheight{\ifdim\Gin@nat@height>\textheight\textheight\else\Gin@nat@height\fi}
\makeatother
% Scale images if necessary, so that they will not overflow the page
% margins by default, and it is still possible to overwrite the defaults
% using explicit options in \includegraphics[width, height, ...]{}
\setkeys{Gin}{width=\maxwidth,height=\maxheight,keepaspectratio}
% Set default figure placement to htbp
\makeatletter
\def\fps@figure{htbp}
\makeatother
\setlength{\emergencystretch}{3em} % prevent overfull lines
\providecommand{\tightlist}{%
  \setlength{\itemsep}{0pt}\setlength{\parskip}{0pt}}
\setcounter{secnumdepth}{-\maxdimen} % remove section numbering

\title{R Notebook: Function}
\author{}
\date{\vspace{-2.5em}}

\begin{document}
\maketitle

\begin{Shaded}
\begin{Highlighting}[]
\NormalTok{f1 <-}\StringTok{ }\ControlFlowTok{function}\NormalTok{(x)\{}
\NormalTok{  result <-}\StringTok{ }\NormalTok{x}\OperatorTok{^}\DecValTok{2-5}
  \KeywordTok{return}\NormalTok{ (result)}
\NormalTok{\}}

\NormalTok{f2 <-}\StringTok{ }\ControlFlowTok{function}\NormalTok{(x)\{}
\NormalTok{  result <-}\StringTok{ }\KeywordTok{sqrt}\NormalTok{(x)}
  \KeywordTok{return}\NormalTok{ (result)}
\NormalTok{\}}
\end{Highlighting}
\end{Shaded}

\begin{Shaded}
\begin{Highlighting}[]
\KeywordTok{print}\NormalTok{(}\KeywordTok{f1}\NormalTok{(}\DecValTok{2}\NormalTok{))}
\end{Highlighting}
\end{Shaded}

\begin{verbatim}
## [1] -1
\end{verbatim}

\begin{Shaded}
\begin{Highlighting}[]
\KeywordTok{print}\NormalTok{(}\KeywordTok{f2}\NormalTok{(}\DecValTok{3}\NormalTok{))}
\end{Highlighting}
\end{Shaded}

\begin{verbatim}
## [1] 1.732051
\end{verbatim}

\hypertarget{exercise-3}{%
\subsection{Exercise 3}\label{exercise-3}}

\hypertarget{section}{%
\subsubsection{1.1}\label{section}}

\begin{Shaded}
\begin{Highlighting}[]
\CommentTok{#1.1}
\NormalTok{f <-}\StringTok{ }\ControlFlowTok{function}\NormalTok{(x) \{}
\NormalTok{  result <-}\StringTok{ }\NormalTok{x}\OperatorTok{^}\DecValTok{3} \OperatorTok{+}\StringTok{ }\NormalTok{x}\OperatorTok{^}\DecValTok{2} \DecValTok{-6}
  \KeywordTok{return}\NormalTok{(result)}
\NormalTok{\}}
\NormalTok{f}
\end{Highlighting}
\end{Shaded}

\begin{verbatim}
## function(x) {
##   result <- x^3 + x^2 -6
##   return(result)
## }
\end{verbatim}

\begin{Shaded}
\begin{Highlighting}[]
\KeywordTok{f}\NormalTok{(}\DecValTok{2}\NormalTok{)}
\end{Highlighting}
\end{Shaded}

\begin{verbatim}
## [1] 6
\end{verbatim}

\hypertarget{section-1}{%
\subsubsection{1.2}\label{section-1}}

\begin{Shaded}
\begin{Highlighting}[]
\CommentTok{#1.2}
\NormalTok{g <-}\StringTok{ }\ControlFlowTok{function}\NormalTok{(a,b) \{}
\NormalTok{  gx <-}\StringTok{ }\NormalTok{a}\OperatorTok{*}\NormalTok{b}\OperatorTok{*}\NormalTok{(b}\OperatorTok{-}\NormalTok{a)}
  \KeywordTok{return}\NormalTok{(gx)}
\NormalTok{\}}
\NormalTok{g}
\end{Highlighting}
\end{Shaded}

\begin{verbatim}
## function(a,b) {
##   gx <- a*b*(b-a)
##   return(gx)
## }
\end{verbatim}

\begin{Shaded}
\begin{Highlighting}[]
\KeywordTok{g}\NormalTok{(}\DecValTok{1}\NormalTok{,}\DecValTok{2}\NormalTok{)}
\end{Highlighting}
\end{Shaded}

\begin{verbatim}
## [1] 2
\end{verbatim}

\hypertarget{section-2}{%
\subsubsection{1.3}\label{section-2}}

\begin{Shaded}
\begin{Highlighting}[]
\CommentTok{#1.3}
\CommentTok{#h(m, n) = (√m/n) + m − 2n}
\NormalTok{h <-}\StringTok{ }\ControlFlowTok{function}\NormalTok{(m, n) \{}
\NormalTok{  hx <-}\StringTok{ }\NormalTok{(}\KeywordTok{sqrt}\NormalTok{(m)}\OperatorTok{/}\NormalTok{n) }\OperatorTok{+}\StringTok{ }\NormalTok{m }\OperatorTok{-}\StringTok{ }\DecValTok{2}\OperatorTok{*}\NormalTok{n}
  \KeywordTok{return}\NormalTok{(hx)}
\NormalTok{\}}
\NormalTok{h}
\end{Highlighting}
\end{Shaded}

\begin{verbatim}
## function(m, n) {
##   hx <- (sqrt(m)/n) + m - 2*n
##   return(hx)
## }
\end{verbatim}

\begin{Shaded}
\begin{Highlighting}[]
\KeywordTok{h}\NormalTok{(}\DecValTok{0}\NormalTok{,}\DecValTok{1}\NormalTok{)}
\end{Highlighting}
\end{Shaded}

\begin{verbatim}
## [1] -2
\end{verbatim}

\hypertarget{section-3}{%
\subsubsection{2.1}\label{section-3}}

\begin{Shaded}
\begin{Highlighting}[]
\CommentTok{#2.1}
\NormalTok{a <-}\StringTok{ }\KeywordTok{matrix}\NormalTok{(}\KeywordTok{c}\NormalTok{(}\DecValTok{1}\OperatorTok{:}\DecValTok{4}\NormalTok{),}\DecValTok{2}\NormalTok{,}\DecValTok{2}\NormalTok{,}\OtherTok{TRUE}\NormalTok{)}
\NormalTok{b <-}\StringTok{ }\KeywordTok{matrix}\NormalTok{(}\KeywordTok{c}\NormalTok{(}\DecValTok{5}\OperatorTok{:}\DecValTok{8}\NormalTok{),}\DecValTok{2}\NormalTok{,}\DecValTok{2}\NormalTok{,}\OtherTok{TRUE}\NormalTok{)}
\NormalTok{a}
\end{Highlighting}
\end{Shaded}

\begin{verbatim}
##      [,1] [,2]
## [1,]    1    2
## [2,]    3    4
\end{verbatim}

\begin{Shaded}
\begin{Highlighting}[]
\NormalTok{b}
\end{Highlighting}
\end{Shaded}

\begin{verbatim}
##      [,1] [,2]
## [1,]    5    6
## [2,]    7    8
\end{verbatim}

\begin{Shaded}
\begin{Highlighting}[]
\CommentTok{#f(a, b) = (a + b)ab}
\NormalTok{f <-}\StringTok{ }\ControlFlowTok{function}\NormalTok{(a, b) \{}
\NormalTok{  fab <-}\StringTok{ }\NormalTok{(a}\OperatorTok{+}\NormalTok{b)}\OperatorTok\NormalTok{a}\OperatorTok\NormalTok{b}
  \KeywordTok{return}\NormalTok{(fab)}
\NormalTok{\}}
\KeywordTok{f}\NormalTok{(a,b)}
\end{Highlighting}
\end{Shaded}

\begin{verbatim}
##      [,1] [,2]
## [1,]  458  532
## [2,]  706  820
\end{verbatim}

\hypertarget{section-4}{%
\subsubsection{2.2}\label{section-4}}

\begin{Shaded}
\begin{Highlighting}[]
\CommentTok{#2.2}
\CommentTok{#MATRIX}
\NormalTok{m <-}\StringTok{ }\KeywordTok{matrix}\NormalTok{(}\KeywordTok{c}\NormalTok{(}\DecValTok{3}\OperatorTok{:}\DecValTok{6}\NormalTok{),}\DecValTok{2}\NormalTok{,}\DecValTok{2}\NormalTok{,}\OtherTok{TRUE}\NormalTok{)}
\NormalTok{n <-}\StringTok{ }\KeywordTok{matrix}\NormalTok{(}\KeywordTok{c}\NormalTok{(}\DecValTok{2}\OperatorTok{:}\DecValTok{5}\NormalTok{),}\DecValTok{2}\NormalTok{,}\DecValTok{2}\NormalTok{,}\OtherTok{TRUE}\NormalTok{)}
\NormalTok{m}
\end{Highlighting}
\end{Shaded}

\begin{verbatim}
##      [,1] [,2]
## [1,]    3    4
## [2,]    5    6
\end{verbatim}

\begin{Shaded}
\begin{Highlighting}[]
\NormalTok{n}
\end{Highlighting}
\end{Shaded}

\begin{verbatim}
##      [,1] [,2]
## [1,]    2    3
## [2,]    4    5
\end{verbatim}

\begin{Shaded}
\begin{Highlighting}[]
\CommentTok{#h(m, n) = |m|n − mn}
\NormalTok{h <-}\StringTok{ }\ControlFlowTok{function}\NormalTok{(m,n)\{}
\NormalTok{  mn <-}\StringTok{ }\KeywordTok{det}\NormalTok{(m)}\OperatorTok{*}\NormalTok{n }\OperatorTok{-}\StringTok{ }\NormalTok{m}\OperatorTok\NormalTok{n}
  \KeywordTok{return}\NormalTok{(mn)}
\NormalTok{\}}
\KeywordTok{h}\NormalTok{(m,n)}
\end{Highlighting}
\end{Shaded}

\begin{verbatim}
##      [,1] [,2]
## [1,]  -26  -35
## [2,]  -42  -55
\end{verbatim}

\hypertarget{section-5}{%
\subsubsection{3.3}\label{section-5}}

\begin{Shaded}
\begin{Highlighting}[]
\CommentTok{#2.3}
\CommentTok{#g(x) = x'x − 2x }
\CommentTok{# x' adalah x transpose}

\NormalTok{x <-}\StringTok{ }\KeywordTok{matrix}\NormalTok{(}\KeywordTok{c}\NormalTok{(}\DecValTok{4}\OperatorTok{:}\DecValTok{7}\NormalTok{),}\DecValTok{2}\NormalTok{,}\DecValTok{2}\NormalTok{,}\OtherTok{TRUE}\NormalTok{)}
\NormalTok{y <-}\StringTok{ }\KeywordTok{t}\NormalTok{(x) }\CommentTok{#transpos dari x}
\NormalTok{x}
\end{Highlighting}
\end{Shaded}

\begin{verbatim}
##      [,1] [,2]
## [1,]    4    5
## [2,]    6    7
\end{verbatim}

\begin{Shaded}
\begin{Highlighting}[]
\NormalTok{y}
\end{Highlighting}
\end{Shaded}

\begin{verbatim}
##      [,1] [,2]
## [1,]    4    6
## [2,]    5    7
\end{verbatim}

\begin{Shaded}
\begin{Highlighting}[]
\NormalTok{g <-}\StringTok{ }\ControlFlowTok{function}\NormalTok{(x) \{}
\NormalTok{  gx <-}\StringTok{ }\KeywordTok{t}\NormalTok{(x)}\OperatorTok{-}\DecValTok{2}\OperatorTok{*}\NormalTok{x}
  \KeywordTok{return}\NormalTok{(gx)}
\NormalTok{\}}
\KeywordTok{g}\NormalTok{(x)}
\end{Highlighting}
\end{Shaded}

\begin{verbatim}
##      [,1] [,2]
## [1,]   -4   -4
## [2,]   -7   -7
\end{verbatim}

\hypertarget{constant-function}{%
\subsection{Constant function}\label{constant-function}}

Reference for plot types can be accessed here
\url{http://www.sthda.com/english/wiki/generic-plot-types-in-r-software}

\begin{Shaded}
\begin{Highlighting}[]
\NormalTok{f <-}\StringTok{ }\ControlFlowTok{function}\NormalTok{(x)\{}
  \CommentTok{#suppose c = 2}
\NormalTok{  fx <-}\StringTok{ }\DecValTok{2}
  \KeywordTok{return}\NormalTok{ (fx)}
\NormalTok{\}}

\NormalTok{input <-}\StringTok{ }\DecValTok{1}\OperatorTok{:}\DecValTok{10}
\KeywordTok{plot}\NormalTok{(input, }\KeywordTok{sapply}\NormalTok{(input, f), }\DataTypeTok{type =} \StringTok{"l"}\NormalTok{, }\DataTypeTok{xlab =} \StringTok{"x"}\NormalTok{, }\DataTypeTok{ylab =} \StringTok{"y"}\NormalTok{)}
\end{Highlighting}
\end{Shaded}

\includegraphics{function-1_files/figure-latex/unnamed-chunk-12-1.pdf}

\hypertarget{linear-function}{%
\subsection{Linear Function}\label{linear-function}}

\begin{Shaded}
\begin{Highlighting}[]
\NormalTok{f <-}\StringTok{ }\ControlFlowTok{function}\NormalTok{(x)\{}
  \CommentTok{#a=1 and b=2}
\NormalTok{  fx <-}\StringTok{ }\DecValTok{1}\OperatorTok{*}\NormalTok{x}\OperatorTok{+}\DecValTok{2}
  \KeywordTok{return}\NormalTok{(fx)}
\NormalTok{\}}

\NormalTok{input <-}\StringTok{ }\DecValTok{-1}\OperatorTok{:}\DecValTok{10}
\KeywordTok{plot}\NormalTok{(input, }\KeywordTok{sapply}\NormalTok{(input, f), }\DataTypeTok{type =} \StringTok{"l"}\NormalTok{, }\DataTypeTok{xlab =} \StringTok{"x"}\NormalTok{, }\DataTypeTok{ylab =} \StringTok{"f(x)"}\NormalTok{)}
\end{Highlighting}
\end{Shaded}

\includegraphics{function-1_files/figure-latex/unnamed-chunk-13-1.pdf}
\#\# Quadratic Function

\begin{Shaded}
\begin{Highlighting}[]
\NormalTok{f <-}\StringTok{ }\ControlFlowTok{function}\NormalTok{(x)\{}
  \CommentTok{#a<0}
\NormalTok{  fx <-}\StringTok{ }\DecValTok{-1}\OperatorTok{*}\NormalTok{x}\OperatorTok{^}\DecValTok{2}\OperatorTok{+}\DecValTok{2}\OperatorTok{*}\NormalTok{x}\OperatorTok{+}\DecValTok{1}
  \KeywordTok{return}\NormalTok{(fx)}
\NormalTok{\}}

\NormalTok{input <-}\StringTok{ }\DecValTok{-20}\OperatorTok{:}\DecValTok{22}
\KeywordTok{plot}\NormalTok{(input, }\KeywordTok{sapply}\NormalTok{(input, f), }\DataTypeTok{type =} \StringTok{"l"}\NormalTok{, }\DataTypeTok{xlab =} \StringTok{"x"}\NormalTok{, }\DataTypeTok{ylab =} \StringTok{"f(x)"}\NormalTok{)}
\end{Highlighting}
\end{Shaded}

\includegraphics{function-1_files/figure-latex/unnamed-chunk-14-1.pdf}
\#\# Polynomial Function

\begin{Shaded}
\begin{Highlighting}[]
\NormalTok{f <-}\StringTok{ }\ControlFlowTok{function}\NormalTok{(x)\{}
\NormalTok{  fx <-}\StringTok{ }\NormalTok{x}\OperatorTok{^}\DecValTok{3-3}\OperatorTok{*}\NormalTok{x}\OperatorTok{^}\DecValTok{2}\OperatorTok{+}\NormalTok{x}\DecValTok{-1}
  \KeywordTok{return}\NormalTok{(fx)}
\NormalTok{\}}

\NormalTok{input <-}\StringTok{ }\KeywordTok{seq}\NormalTok{(}\OperatorTok{-}\DecValTok{10}\NormalTok{, }\DecValTok{11}\NormalTok{, }\FloatTok{0.1}\NormalTok{)}
\KeywordTok{plot}\NormalTok{(input, }\KeywordTok{sapply}\NormalTok{(input, f), }\DataTypeTok{type =} \StringTok{"l"}\NormalTok{, }\DataTypeTok{xlab =} \StringTok{"x"}\NormalTok{, }\DataTypeTok{ylab =} \StringTok{"f(x)"}\NormalTok{)}
\end{Highlighting}
\end{Shaded}

\includegraphics{function-1_files/figure-latex/unnamed-chunk-15-1.pdf}
\#\# Rational Function

\begin{Shaded}
\begin{Highlighting}[]
\NormalTok{f <-}\StringTok{ }\ControlFlowTok{function}\NormalTok{(x)\{}
\NormalTok{  fx <-}\StringTok{ }\DecValTok{1}\OperatorTok{/}\NormalTok{x}
  \KeywordTok{return}\NormalTok{(fx)}
\NormalTok{\}}

\NormalTok{input <-}\StringTok{ }\KeywordTok{seq}\NormalTok{(}\OperatorTok{-}\DecValTok{10}\NormalTok{, }\DecValTok{10}\NormalTok{, }\FloatTok{0.1}\NormalTok{)}
\CommentTok{#to remove 0 from the domain of f}
\NormalTok{input <-}\StringTok{ }\NormalTok{input[}\OperatorTok{-}\KeywordTok{which}\NormalTok{(input}\OperatorTok{==}\DecValTok{0}\NormalTok{)]}
\KeywordTok{plot}\NormalTok{(input, }\KeywordTok{sapply}\NormalTok{(input, f), }\DataTypeTok{type =} \StringTok{"p"}\NormalTok{, }\DataTypeTok{xlab =} \StringTok{"x"}\NormalTok{, }\DataTypeTok{ylab =} \StringTok{"y"}\NormalTok{)}
\end{Highlighting}
\end{Shaded}

\includegraphics{function-1_files/figure-latex/unnamed-chunk-16-1.pdf}

\hypertarget{exercise-4}{%
\subsection{Exercise 4}\label{exercise-4}}

\begin{Shaded}
\begin{Highlighting}[]
\CommentTok{#f(x) = sin(x)}
\NormalTok{f <-}\StringTok{ }\ControlFlowTok{function}\NormalTok{(x) \{}
\NormalTok{  fx <-}\StringTok{ }\KeywordTok{sin}\NormalTok{(x}\OperatorTok{*}\NormalTok{pi}\OperatorTok{/}\DecValTok{180}\NormalTok{)}
  \KeywordTok{return}\NormalTok{(fx)}
\NormalTok{\}}

\NormalTok{input <-}\StringTok{ }\KeywordTok{seq}\NormalTok{(}\OperatorTok{-}\DecValTok{180}\NormalTok{,}\DecValTok{180}\NormalTok{,}\FloatTok{0.1}\NormalTok{)}
\KeywordTok{plot}\NormalTok{(input, }\KeywordTok{sapply}\NormalTok{(input, f), }\DataTypeTok{type =} \StringTok{"l"}\NormalTok{ , }\DataTypeTok{xlab =} \StringTok{"x"}\NormalTok{ , }\DataTypeTok{ylab =} \StringTok{" plot f(x)"}\NormalTok{ )}
\end{Highlighting}
\end{Shaded}

\includegraphics{function-1_files/figure-latex/unnamed-chunk-17-1.pdf}

\begin{Shaded}
\begin{Highlighting}[]
\NormalTok{f <-}\StringTok{ }\ControlFlowTok{function}\NormalTok{(x)\{}
\NormalTok{  fx <-}\StringTok{ }\KeywordTok{log10}\NormalTok{(x)}
  \KeywordTok{return}\NormalTok{ (fx)}
\NormalTok{\}}

\NormalTok{input <-}\StringTok{ }\DecValTok{0}\OperatorTok{:}\DecValTok{100}
\KeywordTok{plot}\NormalTok{(input, }\KeywordTok{sapply}\NormalTok{(input, f), }\DataTypeTok{type =} \StringTok{"p"}\NormalTok{, }\DataTypeTok{xlab =} \StringTok{"x"}\NormalTok{, }\DataTypeTok{ylab =} \StringTok{"f(x)"}\NormalTok{)}
\end{Highlighting}
\end{Shaded}

\includegraphics{function-1_files/figure-latex/unnamed-chunk-18-1.pdf}

\end{document}
