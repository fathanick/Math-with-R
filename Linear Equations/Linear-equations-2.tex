% Options for packages loaded elsewhere
\PassOptionsToPackage{unicode}{hyperref}
\PassOptionsToPackage{hyphens}{url}
%
\documentclass[
]{article}
\usepackage{lmodern}
\usepackage{amssymb,amsmath}
\usepackage{ifxetex,ifluatex}
\ifnum 0\ifxetex 1\fi\ifluatex 1\fi=0 % if pdftex
  \usepackage[T1]{fontenc}
  \usepackage[utf8]{inputenc}
  \usepackage{textcomp} % provide euro and other symbols
\else % if luatex or xetex
  \usepackage{unicode-math}
  \defaultfontfeatures{Scale=MatchLowercase}
  \defaultfontfeatures[\rmfamily]{Ligatures=TeX,Scale=1}
\fi
% Use upquote if available, for straight quotes in verbatim environments
\IfFileExists{upquote.sty}{\usepackage{upquote}}{}
\IfFileExists{microtype.sty}{% use microtype if available
  \usepackage[]{microtype}
  \UseMicrotypeSet[protrusion]{basicmath} % disable protrusion for tt fonts
}{}
\makeatletter
\@ifundefined{KOMAClassName}{% if non-KOMA class
  \IfFileExists{parskip.sty}{%
    \usepackage{parskip}
  }{% else
    \setlength{\parindent}{0pt}
    \setlength{\parskip}{6pt plus 2pt minus 1pt}}
}{% if KOMA class
  \KOMAoptions{parskip=half}}
\makeatother
\usepackage{xcolor}
\IfFileExists{xurl.sty}{\usepackage{xurl}}{} % add URL line breaks if available
\IfFileExists{bookmark.sty}{\usepackage{bookmark}}{\usepackage{hyperref}}
\hypersetup{
  pdftitle={Solving Systems of Linear Equations with Three Unknowns using R},
  hidelinks,
  pdfcreator={LaTeX via pandoc}}
\urlstyle{same} % disable monospaced font for URLs
\usepackage[margin=1in]{geometry}
\usepackage{color}
\usepackage{fancyvrb}
\newcommand{\VerbBar}{|}
\newcommand{\VERB}{\Verb[commandchars=\\\{\}]}
\DefineVerbatimEnvironment{Highlighting}{Verbatim}{commandchars=\\\{\}}
% Add ',fontsize=\small' for more characters per line
\usepackage{framed}
\definecolor{shadecolor}{RGB}{248,248,248}
\newenvironment{Shaded}{\begin{snugshade}}{\end{snugshade}}
\newcommand{\AlertTok}[1]{\textcolor[rgb]{0.94,0.16,0.16}{#1}}
\newcommand{\AnnotationTok}[1]{\textcolor[rgb]{0.56,0.35,0.01}{\textbf{\textit{#1}}}}
\newcommand{\AttributeTok}[1]{\textcolor[rgb]{0.77,0.63,0.00}{#1}}
\newcommand{\BaseNTok}[1]{\textcolor[rgb]{0.00,0.00,0.81}{#1}}
\newcommand{\BuiltInTok}[1]{#1}
\newcommand{\CharTok}[1]{\textcolor[rgb]{0.31,0.60,0.02}{#1}}
\newcommand{\CommentTok}[1]{\textcolor[rgb]{0.56,0.35,0.01}{\textit{#1}}}
\newcommand{\CommentVarTok}[1]{\textcolor[rgb]{0.56,0.35,0.01}{\textbf{\textit{#1}}}}
\newcommand{\ConstantTok}[1]{\textcolor[rgb]{0.00,0.00,0.00}{#1}}
\newcommand{\ControlFlowTok}[1]{\textcolor[rgb]{0.13,0.29,0.53}{\textbf{#1}}}
\newcommand{\DataTypeTok}[1]{\textcolor[rgb]{0.13,0.29,0.53}{#1}}
\newcommand{\DecValTok}[1]{\textcolor[rgb]{0.00,0.00,0.81}{#1}}
\newcommand{\DocumentationTok}[1]{\textcolor[rgb]{0.56,0.35,0.01}{\textbf{\textit{#1}}}}
\newcommand{\ErrorTok}[1]{\textcolor[rgb]{0.64,0.00,0.00}{\textbf{#1}}}
\newcommand{\ExtensionTok}[1]{#1}
\newcommand{\FloatTok}[1]{\textcolor[rgb]{0.00,0.00,0.81}{#1}}
\newcommand{\FunctionTok}[1]{\textcolor[rgb]{0.00,0.00,0.00}{#1}}
\newcommand{\ImportTok}[1]{#1}
\newcommand{\InformationTok}[1]{\textcolor[rgb]{0.56,0.35,0.01}{\textbf{\textit{#1}}}}
\newcommand{\KeywordTok}[1]{\textcolor[rgb]{0.13,0.29,0.53}{\textbf{#1}}}
\newcommand{\NormalTok}[1]{#1}
\newcommand{\OperatorTok}[1]{\textcolor[rgb]{0.81,0.36,0.00}{\textbf{#1}}}
\newcommand{\OtherTok}[1]{\textcolor[rgb]{0.56,0.35,0.01}{#1}}
\newcommand{\PreprocessorTok}[1]{\textcolor[rgb]{0.56,0.35,0.01}{\textit{#1}}}
\newcommand{\RegionMarkerTok}[1]{#1}
\newcommand{\SpecialCharTok}[1]{\textcolor[rgb]{0.00,0.00,0.00}{#1}}
\newcommand{\SpecialStringTok}[1]{\textcolor[rgb]{0.31,0.60,0.02}{#1}}
\newcommand{\StringTok}[1]{\textcolor[rgb]{0.31,0.60,0.02}{#1}}
\newcommand{\VariableTok}[1]{\textcolor[rgb]{0.00,0.00,0.00}{#1}}
\newcommand{\VerbatimStringTok}[1]{\textcolor[rgb]{0.31,0.60,0.02}{#1}}
\newcommand{\WarningTok}[1]{\textcolor[rgb]{0.56,0.35,0.01}{\textbf{\textit{#1}}}}
\usepackage{graphicx,grffile}
\makeatletter
\def\maxwidth{\ifdim\Gin@nat@width>\linewidth\linewidth\else\Gin@nat@width\fi}
\def\maxheight{\ifdim\Gin@nat@height>\textheight\textheight\else\Gin@nat@height\fi}
\makeatother
% Scale images if necessary, so that they will not overflow the page
% margins by default, and it is still possible to overwrite the defaults
% using explicit options in \includegraphics[width, height, ...]{}
\setkeys{Gin}{width=\maxwidth,height=\maxheight,keepaspectratio}
% Set default figure placement to htbp
\makeatletter
\def\fps@figure{htbp}
\makeatother
\setlength{\emergencystretch}{3em} % prevent overfull lines
\providecommand{\tightlist}{%
  \setlength{\itemsep}{0pt}\setlength{\parskip}{0pt}}
\setcounter{secnumdepth}{-\maxdimen} % remove section numbering

\title{Solving Systems of Linear Equations with Three Unknowns using R}
\author{}
\date{\vspace{-2.5em}}

\begin{document}
\maketitle

\hypertarget{equations-in-three-unknowns}{%
\subsubsection{Equations in three
unknowns}\label{equations-in-three-unknowns}}

\begin{Shaded}
\begin{Highlighting}[]
\KeywordTok{library}\NormalTok{(matlib)}
\end{Highlighting}
\end{Shaded}

Given a system of equations,

\[
\begin{alignat*}{7}
2x &&\; + \;&& y  &&\; -z &&\; = \;&& 8  \\
-3x &&\; - \;&& y &&\; +2z  &&\; = \;&& -11 \\
-2x &&\; + \;&& y &&\; +2z &&\; = \;&& -3
\end{alignat*}
\]\\
And here is the augmented matrix

\[
\begin{bmatrix}
    2 & 1 & -1 \\
  -3 & -1 & 2 \\
  -2 & 1 & 2
  \end{bmatrix}
    \begin{bmatrix}
      x_{1} \\
    x_{2} \\
    x_{3}
\end{bmatrix}
    =
    \begin{bmatrix}
    8 \\
    -11\\
    -3
    \end{bmatrix}
\] \#\#\#\# Create and show the equation

\begin{Shaded}
\begin{Highlighting}[]
\NormalTok{A <-}\StringTok{ }\KeywordTok{matrix}\NormalTok{(}\KeywordTok{c}\NormalTok{(}\DecValTok{2}\NormalTok{,}\DecValTok{1}\NormalTok{,}\OperatorTok{-}\DecValTok{1}\NormalTok{,}\OperatorTok{-}\DecValTok{3}\NormalTok{,}\OperatorTok{-}\DecValTok{1}\NormalTok{,}\DecValTok{2}\NormalTok{,}\OperatorTok{-}\DecValTok{2}\NormalTok{,}\DecValTok{1}\NormalTok{,}\DecValTok{2}\NormalTok{),}\DecValTok{3}\NormalTok{,}\DecValTok{3}\NormalTok{, }\OtherTok{TRUE}\NormalTok{)}
\NormalTok{b <-}\StringTok{ }\KeywordTok{c}\NormalTok{(}\DecValTok{8}\NormalTok{,}\OperatorTok{-}\DecValTok{11}\NormalTok{,}\OperatorTok{-}\DecValTok{3}\NormalTok{)}
\KeywordTok{showEqn}\NormalTok{(A,b)}
\end{Highlighting}
\end{Shaded}

\begin{verbatim}
##  2*x1 + 1*x2 - 1*x3  =    8 
## -3*x1 - 1*x2 + 2*x3  =  -11 
## -2*x1 + 1*x2 + 2*x3  =   -3
\end{verbatim}

\hypertarget{find-solution}{%
\paragraph{Find solution}\label{find-solution}}

\begin{Shaded}
\begin{Highlighting}[]
\KeywordTok{Solve}\NormalTok{(A,b)}
\end{Highlighting}
\end{Shaded}

\begin{verbatim}
## x1      =   2 
##   x2    =   3 
##     x3  =  -1
\end{verbatim}

\hypertarget{plot-the-equation}{%
\paragraph{Plot the equation}\label{plot-the-equation}}

\begin{Shaded}
\begin{Highlighting}[]
\KeywordTok{plotEqn3d}\NormalTok{(A,b)}
\end{Highlighting}
\end{Shaded}

\hypertarget{reduced-echelon-form}{%
\paragraph{Reduced echelon form}\label{reduced-echelon-form}}

\begin{Shaded}
\begin{Highlighting}[]
\KeywordTok{echelon}\NormalTok{(A, b, }\DataTypeTok{verbose=}\OtherTok{TRUE}\NormalTok{, }\DataTypeTok{fractions=}\OtherTok{TRUE}\NormalTok{)}
\end{Highlighting}
\end{Shaded}

\begin{verbatim}
## 
## Initial matrix:
##      [,1] [,2] [,3] [,4]
## [1,]   2    1   -1    8 
## [2,]  -3   -1    2  -11 
## [3,]  -2    1    2   -3 
## 
## row: 1 
## 
##  exchange rows 1 and 2 
##      [,1] [,2] [,3] [,4]
## [1,]  -3   -1    2  -11 
## [2,]   2    1   -1    8 
## [3,]  -2    1    2   -3 
## 
##  multiply row 1 by -1/3 
##      [,1] [,2] [,3] [,4]
## [1,]    1  1/3 -2/3 11/3
## [2,]    2    1   -1    8
## [3,]   -2    1    2   -3
## 
##  multiply row 1 by 2 and subtract from row 2 
##      [,1] [,2] [,3] [,4]
## [1,]    1  1/3 -2/3 11/3
## [2,]    0  1/3  1/3  2/3
## [3,]   -2    1    2   -3
## 
##  multiply row 1 by 2 and add to row 3 
##      [,1] [,2] [,3] [,4]
## [1,]    1  1/3 -2/3 11/3
## [2,]    0  1/3  1/3  2/3
## [3,]    0  5/3  2/3 13/3
## 
## row: 2 
## 
##  exchange rows 2 and 3 
##      [,1] [,2] [,3] [,4]
## [1,]    1  1/3 -2/3 11/3
## [2,]    0  5/3  2/3 13/3
## [3,]    0  1/3  1/3  2/3
## 
##  multiply row 2 by 3/5 
##      [,1] [,2] [,3] [,4]
## [1,]    1  1/3 -2/3 11/3
## [2,]    0    1  2/5 13/5
## [3,]    0  1/3  1/3  2/3
## 
##  multiply row 2 by 1/3 and subtract from row 1 
##      [,1] [,2] [,3] [,4]
## [1,]    1    0 -4/5 14/5
## [2,]    0    1  2/5 13/5
## [3,]    0  1/3  1/3  2/3
## 
##  multiply row 2 by 1/3 and subtract from row 3 
##      [,1] [,2] [,3] [,4]
## [1,]    1    0 -4/5 14/5
## [2,]    0    1  2/5 13/5
## [3,]    0    0  1/5 -1/5
## 
## row: 3 
## 
##  multiply row 3 by 5 
##      [,1] [,2] [,3] [,4]
## [1,]    1    0 -4/5 14/5
## [2,]    0    1  2/5 13/5
## [3,]    0    0    1   -1
## 
##  multiply row 3 by 4/5 and add to row 1 
##      [,1] [,2] [,3] [,4]
## [1,]    1    0    0    2
## [2,]    0    1  2/5 13/5
## [3,]    0    0    1   -1
## 
##  multiply row 3 by 2/5 and subtract from row 2 
##      [,1] [,2] [,3] [,4]
## [1,]  1    0    0    2  
## [2,]  0    1    0    3  
## [3,]  0    0    1   -1
\end{verbatim}

\end{document}
